\chapter{Control de calidad}
Introduzca en esta sección la estrategia utilizada para el control de calidad del producto. Incluya los miembros del equipo que participaron del proceso de control de calidad.
\section{Pruebas Internas}
Introduzca esta sección indicando cuántos scripts de pruebas se hicieron para el sistema propio. Indique cualquier detalle que crea necesario para que se entiendan los scripts de pruebas correctamente. Recuerde referencias los diferentes scripts usando las etiquetas y el \ref{TestSctipt1} correspondiente

\subsection{Guiones de pruebas internas}
\begin{longtable}{ | p{2cm} | p{3cm} | p{4cm} | p{4cm} | c |}
      \hline
      \textbf{Historias} & \textbf{Descripción} & \textbf{Resultado Esperado} & \textbf{Resultado Obtenido} & \textbf{Condición}\\
      \hline
      RF001 y RF003 & AAAA & BBBB & CCCC & \color{ForestGreen}PASÓ\\
      \hline
      RF002 y RF005 & AAAA & BBBB & CCCC & \color{Mahogany}FALLÓ\\
      \hline
      
      \caption{Descripción rápida del guión de pruebas}
      \label{TestScript1}
\end{longtable}

\begin{longtable}{ | p{2cm} | p{3cm} | p{4cm} | p{4cm} | c |}
      \hline
      \textbf{Historias} & \textbf{Descripción} & \textbf{Resultado Esperado} & \textbf{Resultado Obtenido} & \textbf{Condición}\\
      \hline
      RF001 y RF003 & AAAA & BBBB & CCCC & \color{ForestGreen}PASÓ\\
      \hline
      RF002 y RF005 & AAAA & BBBB & CCCC & \color{Mahogany}FALLÓ\\
      \hline
      
      \caption{Descripción rápida del otro guión de pruebas. Note que pueden haber varios guiones para la misma iteración.}
      \label{TestScript1}
\end{longtable}

\section{Pruebas Externas}
Introduzca en esta sección los guiones de pruebas que el otro equipo le aplicó a su sistema. Describa cada guión de pruebas y resuma el resultado general de las pruebas de su sistema. 

\subsection{Guiones de pruebas externas}
Incluya aquí los guiones de pruebas externas. Note que los guiones que debe incorporar son los que se aplicaron a su sistema. Estos debieron haberse entregado por el equipo de control cruzado oportunamente.

\begin{longtable}{ | p{2cm} | p{3cm} | p{4cm} | p{4cm} | c |}
      \hline
      \textbf{Historias} & \textbf{Descripción} & \textbf{Resultado Esperado} & \textbf{Resultado Obtenido} & \textbf{Condición}\\
      \hline
      RF001 y RF003 & AAAA & BBBB & CCCC & \color{ForestGreen}PASÓ\\
      \hline
      RF002 y RF005 & AAAA & BBBB & CCCC & \color{Mahogany}FALLÓ\\
      \hline
      
      \caption{Descripción rápida del guión de pruebas}
      \label{TestScript1}
\end{longtable}