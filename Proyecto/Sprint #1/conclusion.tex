\chapter{Conclusiones y trabajo futuro}
\section{Conclusiones}
\paragraph{}WRITE HERE.
\section{Problemáticas y limitaciones}
Recuerde que sólo debe registrar aquellas problemáticas que el sistema TIENE, no debe mencionar problemas que aparecieron durante el desarrollo, pero que se resolvieron adecuadamente.
Considere como problemáticas errores o comportamientos indeseables que el sistema tiene. Por ejemplo fugas de memoria, puntos en los que el sistema se cuelga, etc...
Limitaciones por el otro lado, son cosas que el sistema NO puede hacer, ya sea por diseño o por que se quería incluir pero no fue posible. Por ejemplo la falta de multijugador por Internet es una limitación, también liste aquí las plataformas para las cuales el sistema está diseñado (sistema operativo, requerimientos mínimos, resolución recomendada, etc...).

\section{Trabajo futuro}
Describa en esta sección qué considera que puede mejorarse en el futuro para este proyecto. 


%%Texto explicativo, no debe aparecer en el documento final
\section{NOTAS SOBRE LAS REFERENCIAS BIBLIOGRÁFICAS}
En el archivo references.bib puede agregar las fuentes bibliográficas usadas en el documento en formato \emph{bibtex}.
Puede incluir fuentes que no citó en el archivo de referencias. Aquellas que no citó automáticamente serán omitidas en la lista de referencias.
Para citar una referencia en el texto debe usar el comando \emph{cite}. 

Por ejemplo el artículo de Dean se cita usando \cite{dean2003}. 
Note cómo los libros de Ortíz y Sánchez no salen en las referencias. 

\emph{Procure eliminar esta sección de su documento}
%%Aquí termina el texto explicativo