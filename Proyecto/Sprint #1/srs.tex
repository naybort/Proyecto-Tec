\chapter{Especificación de requerimientos}
\section{Pila de producto}
Describa brevemente el producto en función del análisis de historias. Si es posible modifique la codificación de las historias para que el lector tenga un apoyo visual respecto a la estructura del sistema. Por ejemplo las historias correspondientes a los módulos de interfaz podrían codificarse UIXXX, mientras que las historias correspondientes a módulos de red podrían codificarse NTXXX. Explique en esta sección la lógica detrás de la codificación usada.

\subsection{Requerimientos funcionales del sistema}

%% PILA DE PRODUCTO %%
\begin{longtable}{|l||p{7cm}|l|l|}

\multicolumn{4}{c}{Pila general del producto}\\
\hline\hline
\textbf{Código} & \textbf{Descripción} & \textbf{Prioridad} & \textbf{Inserción}\\
\hline
\endfirsthead
\textbf{Código} & \textbf{Descripción} & \textbf{Prioridad} & \textbf{Inserción}\\
\hline\hline
\endhead

\textbf{RFXXX} & 
Descripción de la funcionalidad que representa la historia, lo más detallado posible & Alta & Original\\

\textbf{RF001} & Una historia original & Alta & Original \\

\color{ForestGreen}
\textbf{RF002} & Una historia nueva en este sprint & Media & Sprint 1 \\

\color{Mahogany}
\textbf{RF003} & Una historia eliminada & Baja & Original \\
\hline

\caption{\color{ForestGreen}VERDE: Historias agregadas en esta iteración. \color{Mahogany}ROJO: Historias eliminadas.}
\label{ProductBacklog}

\end{longtable}

\subsection{Bitácora de cambios}
Debe resumir en esta sección las decisiones que llevaron a la creación de nuevas historias, o la eliminación de historias ya analizadas. 

\section{Producto Mínimo Viable de la iteración X}

\subsection{Alcance del PMV}
Describa en esta sección el alcance que se proyectó para el PMV de esta iteración. Tome en cuenta que los PMV son incrementales, por lo que debe explicar cómo esta iteración aumentó la funcionalidad del PMV pasado.

\subsection{Pila de trabajo de la iteración X}

%% PILA DEL SPRINT (ITERACIÓN) %%

\begin{longtable}{|l||c|c|p{7cm}|l|c|}

\multicolumn{5}{c}{Pila de la \textbf{iteración X} }\\
\hline\hline
\textbf{Código} & \textbf{CE} & \textbf{CR} & \textbf{Responsables} & \textbf{Finalización}& \\
\hline
\endfirsthead
\textbf{Código} & \textbf{CE} & \textbf{CR} & \textbf{Responsables} & \textbf{Finalización} & \\
\hline\hline
\endhead

\textbf{RF001} & 3 & 3 & Responsable A, Responsable B & 2017/08/14 & $\square$\\

\color{ForestGreen}
\textbf{RF002} & 2 & 4 & Responsable A, Responsable B & 2017/08/25 & $\square$\\

% IMPORTANTE: La casilla de verificación para el Product Owner no debe incluirse en las historias eliminadas ni en las historias inconclusas. Solamente en aquellas que se lograron terminar en el Sprint
\color{Mahogany}
\textbf{RF003} & 5 & - & Responsable A, Responsable B & N/A &  \\
\hline

\caption{Pila de la Iteración X. \textbf{CE:} Carga Estimada, \textbf{CR:} Carga Real.}
\label{SprintBacklog}


\end{longtable}
